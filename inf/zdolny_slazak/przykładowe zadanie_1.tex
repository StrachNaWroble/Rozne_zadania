\documentclass[a4paper,11pt]{article}

% Polskie znaki i kodowanie
\usepackage[utf8]{inputenc}
\usepackage[T1]{fontenc}
\usepackage[polish]{babel}
\usepackage{polski}

% Pakiety do układu strony i grafiki
\usepackage{geometry}
\geometry{
 a4paper,
 total={170mm,257mm},
 left=20mm,
 top=20mm,
}
\usepackage{multicol}
\usepackage{xcolor}
\usepackage{listings}
\usepackage{enumitem}
\usepackage{amssymb}

% Konfiguracja wyglądu kodu (listings)
\definecolor{codegreen}{rgb}{0,0.6,0}
\definecolor{codegray}{rgb}{0.5,0.5,0.5}
\definecolor{codepurple}{rgb}{0.58,0,0.82}
\definecolor{backcolour}{rgb}{0.95,0.95,0.92}

\lstset{
    backgroundcolor=\color{backcolour},   
    commentstyle=\color{codegreen},
    keywordstyle=\color{blue},
    numberstyle=\tiny\color{codegray},
    stringstyle=\color{codepurple},
    basicstyle=\ttfamily\footnotesize,
    breakatwhitespace=false,         
    breaklines=true,                 
    captionpos=b,                    
    keepspaces=true,                 
    numbers=left,                    
    numbersep=5pt,                  
    showspaces=false,                
    showstringspaces=false,
    showtabs=false,                  
    tabsize=2,
    frame=single
}

% Definicja wyglądu pytania (symbol ? na początku)
\newcommand{\pytanie}[1]{
    \vspace{0.3cm}
    \noindent \textbf{? #1}
}

\title{Przykładowe zadanie 1 - etap powiatowy}
\author{Justyna Rojek-Nowosielska}
\date{}

\begin{document}

\maketitle
\vspace{1.5cm}

\noindent \textbf{Wstęp} \\
Rozważmy następujący program. Program ten realizuje algorytm sprawdzający pewną własność liczb naturalnych.

\vspace{0.5cm}

% Sekcja z kodem C++ i Python obok siebie
\begin{minipage}[t]{0.48\textwidth}
\textbf{C++}
\begin{lstlisting}[language=C++]
#include <iostream>
#include <vector>
using namespace std;

int main() {
    int n;
    cin >> n;
    vector<int> wynik;
    for (int i = 1; i <= n; i++) {
        int suma_dzielnikow = 0;
        for (int j = 1; j < i; j++) {
            if (i % j == 0) {
                suma_dzielnikow += j;
            }
        }
        if (suma_dzielnikow > i) {
            wynik.push_back(i);
        }
    }
    for (int liczba : wynik) {
        cout << liczba << " ";
    }
    return 0;
}
\end{lstlisting}
\end{minipage}
\hfill

\vspace{0.5cm}
\hrule
\vspace{0.5cm}


\noindent Program przyjmuje na wejściu dodatnią liczbę całkowitą $n$. 
Przykładowo: dla liczby $n = 15$ program wypisze tylko liczbę 12.

% Pytania
\pytanie{1.1. Co oblicza powyższy program?}
\begin{enumerate}[label=(\alph*), noitemsep]
    \item zbiór dzielników liczby n
    \item zbiór sum liczb pierwszych mniejszych od n
    \item zbiór liczb, których suma dzielników jest większa od niej samej i które są mniejszse lub równe n
    \item zbiór liczb, których suma cyfr jest większa od samej liczby i które są mniejszse lub równe n
\end{enumerate}

\pytanie{1.2. Co wypisze powyższy program dla n = 20?}

\pytanie{1.3. Podaj najmniejszą dodatnią liczbę całkowitą n, dla której program wypisze dokładnie jedną liczbę.}

\pytanie{1.4. Czy program kiedykolwiek wypisze liczbę pierwszą?}

\pytanie{1.5. Ile jest dodatnich liczb całkowitych, które podane na wejściu (jako n) spowodują, że ostatnią wypisaną liczbą będzie 24?}

\pytanie{1.6. Ile liczb wypisze powyższy program dla danej 50?}

\pytanie{1.7. Ile wynosi suma liczb, które wypisze powyższy program dla danej 50?}

\pytanie{1.8. Podaj najmniejszą liczbę całkowitą n, która podana na wejściu spowoduje wypisanie ciągu o długości co najmniej 5.}

\pytanie{1.9. Ile wynosi suma cyfr najmniejszej liczby nieparzystej, którą mógłby wypisać ten program?} \\

\pytanie{1.10. Złożoność i optymalizacja.} \\
Oryginalny program dla $n = 100\,000$ działałby bardzo wolno. Zauważ, że wewnętrzna pętla działa $i$ razy. Jaką zmianę w algorytmie obliczania sumy dzielników należy wprowadzić, aby program działał znacznie szybciej (w złożoności zbliżonej do $O(n\sqrt{n})$ zamiast $O(n^2)$)?

\end{document}
