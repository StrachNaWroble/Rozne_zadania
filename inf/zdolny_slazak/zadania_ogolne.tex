\documentclass[a4paper, 12pt]{article}
\usepackage{lipsum}
\usepackage{polski}
\usepackage{gensymb}
\usepackage[utf8]{inputenc}
\usepackage[T1]{fontenc}
\usepackage[polish]{babel}
\usepackage{amsmath}
\usepackage{amssymb} 
\usepackage{enumitem} 

\usepackage[letterpaper,top=1mm,bottom=1.5cm,left=2cm,right=2cm,marginparwidth=1.75cm]{geometry}

\usepackage{graphicx}
\usepackage[colorlinks=true, allcolors=blue]{hyperref}

 
\title{Lista 1 - zDolny Ślązak z Informatyki}
\author{Justyna Rojek-Nowosielska}

\renewcommand{\mod}{\ \text{mod}\ }

\begin{document}
\maketitle
\vspace*{1pt}

\subsection*{1. Pozycyjne systemy liczbowe}

\begin{enumerate}
    \item \textbf{System dwójkowy:}
    \begin{enumerate}[label=\alph*)]
        \item Zamień podane liczby (zapisane w systemie dziesiętnym) na system dwójkowy: 13, 42, 129.
        \item Zamień podane liczby (zapisane w systemie dwójkowym) na system dziesiętny: $1011_2$, $110101_2$, $10000001_2$.
    \end{enumerate}

    \item \textbf{System ósemkowy:}
    \begin{enumerate}[label=\alph*)]
        \item Zamień podane liczby (zapisane w systemie dziesiętnym) na system ósemkowy: 9, 70, 513.
        \item Zamień podane liczby (zapisane w systemie ósemkowym) na system dziesiętny: $15_8$, $101_8$, $273_8$.
    \end{enumerate}

    \item \textbf{System szesnastkowy:}
    \begin{enumerate}[label=\alph*)]
        \item Zamień podane liczby (zapisane w systemie dziesiętnym) na system szesnastkowy: 20, 255, 400.
        \item Zamień podane liczby (zapisane w systemie szesnastkowym) na system dziesiętny: $1A_{16}$, $F0_{16}$, $101_{16}$.
    \end{enumerate}
    
    \item \textbf{Zadanie dodatkowe:} Uporządkuj podane liczby rosnąco (przeliczając je na system dziesiętny): 
    $$100_{16}, \quad 10010011_2, \quad 200_{10}, \quad 300_8$$
\end{enumerate}

\subsection*{2. Cechy podzielności i operacja reszty z dzielenia}

\begin{enumerate}
    \item Oblicz resztę z dzielenia:
    \begin{enumerate}[label=\alph*)]
        \item $(120 + 54) \mod 10$
        \item $(19 \cdot 8) \mod 7$
        \item $5^3 \mod 4$
    \end{enumerate}
    
    \item Pociąg ma 7 wagonów (numerowanych 0, 1, 2, 3, 4, 5, 6). Konduktor jest w wagonie nr 1 i przechodzi 10 wagonów do przodu (z wagonu 6 przechodzi spowrotem do wagonu 0). W którym wagonie się znajdzie? A po 100 przejściach?
    
    \item Jaka jest suma cyfr liczby 12345? Jaki będzie wynik operacji $12345 \% 9$? Jaka jest suma cyfr liczby 554? Jaki będzie wynik operacji $554 \% 9$? (Jak to zrobić, żeby się nie naliczyć za bardzo?)
    
    \item Program oblicza iloczyn cyfr liczby. Jaki będzie wynik dla $n = 105$, a jaki dla $n = 239$?
\end{enumerate}

\break
\subsection*{3. Liczby pierwsze}

\begin{enumerate}
    \item Jaka jest ósma liczba pierwsza?
    
    \item Podaj rozkład na czynniki pierwsze liczb:
    \begin{enumerate}[label=\alph*)]
        \item 120
        \item 300
        \item 512
    \end{enumerate}
    
    \item Ile \textbf{różnych} czynników pierwszych mają liczby:
    \begin{enumerate}[label=\alph*)]
        \item 60 (czynniki to 2, 3, 5, więc 3 różne czynniki)
        \item 100
        \item 48
    \end{enumerate}
    
    \item Jaka jest największa liczba pierwsza mniejsza niż 100?
    
    \item Używając algorytmu Sita Eratostenesa do znalezienia liczb pierwszych w zakresie $[2, 24]$ (wypisz te liczby i skreślaj je po kolei), wielokrotności jakiej liczby pierwszej będą wykreślane jako ostatnie?
\end{enumerate}

\subsection*{4. NWD i NWW}

\begin{enumerate}
    \item Oblicz Największy Wspólny Dzielnik (NWD) dla podanych \textbf{trzech} liczb:
    \begin{enumerate}[label=\alph*)]
        \item 12, 18, 30
        \item 45, 75, 120
    \end{enumerate}
    
    \item Oblicz Najmniejszą Wspólną Wielokrotność (NWW) dla podanych \textbf{trzech} liczb:
    \begin{enumerate}[label=\alph*)]
        \item 3, 5, 6
        \item 8, 12, 16
    \end{enumerate}
    
    \item NWD dwóch liczb $a$ i $b$ wynosi 10, a ich NWW wynosi 120. Jedna z liczb to 30. Jaka jest druga liczba? (Wskazówka: $a \cdot b = NWD(a, b) \cdot NWW(a, b)$).
    
    \item Do jakiej najprostszej (nieskracalnej) postaci można uprościć ułamek $\frac{60}{144}$?
    
    \item Śledzimy kroki algorytmu Euklidesa (w wersji z odejmowaniem) dla liczb 42 i 18. Jaka para liczb będzie przetwarzana tuż przed otrzymaniem wyniku?
\end{enumerate}

\end{document}
