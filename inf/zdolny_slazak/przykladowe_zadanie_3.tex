\documentclass[a4paper,11pt]{article}

% Polskie znaki i kodowanie
\usepackage[utf8]{inputenc}
\usepackage[T1]{fontenc}
\usepackage[polish]{babel}
\usepackage{polski}
% Pakiety do układu strony i grafiki
\usepackage{geometry}
\geometry{
 a4paper,
 total={170mm,257mm},
 left=20mm,
 top=20mm,
}
\usepackage{multicol}
\usepackage{xcolor}
\usepackage{listings}
\usepackage{enumitem}
\usepackage{amssymb}
\usepackage{amsmath}

% Konfiguracja wyglądu kodu
\definecolor{codegreen}{rgb}{0,0.6,0}
\definecolor{codegray}{rgb}{0.5,0.5,0.5}
\definecolor{codepurple}{rgb}{0.58,0,0.82}
\definecolor{backcolour}{rgb}{0.96,0.96,0.96}

\lstset{
    backgroundcolor=\color{backcolour},   
    commentstyle=\color{codegreen},
    keywordstyle=\color{blue},
    numberstyle=\tiny\color{codegray},
    stringstyle=\color{codepurple},
    basicstyle=\ttfamily\footnotesize,
    breakatwhitespace=false,         
    breaklines=true,                 
    captionpos=b,                    
    keepspaces=true,                 
    numbers=left,                    
    numbersep=5pt,                  
    showspaces=false,                
    showstringspaces=false,
    showtabs=false,                  
    tabsize=2,
    frame=single
}

\newcommand{\pytanie}[1]{
    \vspace{0.4cm}
    \noindent \textbf{? #1}
}

\title{Przykładowe zadanie 3 - etap powiatowy}
\date{}
\author{Justyna Rojek-Nowosielska}

\begin{document}

\maketitle
\vspace{-1.5cm}

\noindent \textbf{Wstęp} \\
Rozważmy następujący program, który analizuje ciąg liczb całkowitych dodatnich (możesz traktować je jako ceny pewnego towaru w kolejnych dniach).

\vspace{0.5cm}

% Kod programu
\begin{minipage}[t]{0.48\textwidth}
\textbf{C++}
\begin{lstlisting}[language=C++]
#include <iostream>
#include <vector>
#include <algorithm>
using namespace std;

int main() {
    int n;
    cin >> n;
    vector<int> C(n);
    for (int i = 0; i < n; i++) {
        cin >> C[i];
    }

    int wynik = 0;
    
    for (int j = 1; j < n; j++) {
        for (int i = 0; i < j; i++) {
            int roznica = C[j] - C[i];
            if (roznica > wynik) {
                wynik = roznica;
            }
        }
    }

    cout << wynik << endl;
    return 0;
}
\end{lstlisting}
\end{minipage}

\vspace{0.5cm}
\hrule
\vspace{0.5cm}

\noindent Program przyjmuje na wejściu liczbę $n$ (liczba dni), a następnie $n$ liczb całkowitych oznaczających wartości w kolejne dni.

% Pytania
\pytanie{1.1. Co oblicza powyższy program?}
\begin{enumerate}[label=(\alph*), noitemsep]
    \item największą liczbę występującą w ciągu $C$,
    \item największą różnicę między dwoma sąsiednimi elementami ciągu,
    \item największy możliwy zysk, kupując w dniu $i$ i sprzedając w dniu $j$ (gdzie $j > i$),
    \item liczbę par $(i, j)$ takich, że $C[j] > C[i]$.
\end{enumerate}

\pytanie{1.2. Co wypisze program dla danych wejściowych: \texttt{5} oraz ciągu \texttt{7 1 5 3 6}?}

\pytanie{1.3. Podaj przykład 4-elementowego ciągu, dla którego program wypisze \texttt{0}.}

\pytanie{1.4. Ile operacji odejmowania wykona ten program dla $n=100$? (Przybliżona wartość lub dokładny wzór).}

\pytanie{1.5. Czy wynik działania programu może być liczbą ujemną przy założeniu, że wszystkie liczby na wejściu są dodatnie? Uzasadnij.}

\pytanie{1.6. Jeśli na wejściu podamy ciąg posortowany rosnąco (np. 1, 2, 3, ..., 100), jaki będzie wynik?}
\begin{enumerate}[label=(\alph*), noitemsep]
    \item 0
    \item 100
    \item 99
    \item 5050
\end{enumerate}

\pytanie{1.7. Ile jest par liczb $(i, j)$, które program sprawdza dla ciągu o długości $n=5$?}

\pytanie{1.8. Dla jakiego typu danych wejściowych (układ liczb) ten algorytm zwróci największą możliwą różnicę (równą różnicy między max i min całego ciągu)?}

\pytanie{1.9. Zmodyfikuj (w wyobraźni) jedną linię kodu tak, aby program obliczał największą stratę (czyli największy spadek wartości między dniem $i$ a $j$, gdzie $j > i$).}

\pytanie{1.10. Optymalizacja.} \\
Podany algorytm ma złożoność $O(n^2)$. Dla $n=100\,000$ nie zakończy się w czasie 1 sekundy. 
Zauważ, że aby zmaksymalizować różnicę $C[j] - C[i]$ dla ustalonego $j$, musimy odjąć najmniejszą wartość $C[i]$ napotkaną \textbf{wcześniej}. 
Zaproponuj algorytm (słownie lub kod), który rozwiąże to zadanie, przeglądając tablicę tylko raz ($O(n)$).

\newpage

\end{document}
