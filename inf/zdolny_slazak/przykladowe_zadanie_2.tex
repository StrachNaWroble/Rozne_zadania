\documentclass[a4paper,11pt]{article}

% Polskie znaki i kodowanie
\usepackage[utf8]{inputenc}
\usepackage[T1]{fontenc}
\usepackage[polish]{babel}
\usepackage{polski}

% Pakiety do układu strony i grafiki
\usepackage{geometry}
\geometry{
 a4paper,
 total={170mm,257mm},
 left=20mm,
 top=20mm,
}
\usepackage{multicol}
\usepackage{xcolor}
\usepackage{listings}
\usepackage{enumitem}
\usepackage{amssymb}
\usepackage{amsmath}

% Konfiguracja wyglądu kodu
\definecolor{codegreen}{rgb}{0,0.6,0}
\definecolor{codegray}{rgb}{0.5,0.5,0.5}
\definecolor{codepurple}{rgb}{0.58,0,0.82}
\definecolor{backcolour}{rgb}{0.97,0.97,0.97}

\lstset{
    backgroundcolor=\color{backcolour},   
    commentstyle=\color{codegreen},
    keywordstyle=\color{blue},
    numberstyle=\tiny\color{codegray},
    stringstyle=\color{codepurple},
    basicstyle=\ttfamily\footnotesize,
    breakatwhitespace=false,         
    breaklines=true,                 
    captionpos=b,                    
    keepspaces=true,                 
    numbers=left,                    
    numbersep=5pt,                  
    showspaces=false,                
    showstringspaces=false,
    showtabs=false,                  
    tabsize=2,
    frame=single
}

\newcommand{\pytanie}[1]{
    \vspace{0.4cm}
    \noindent \textbf{? #1}
}

\title{Przykładowe zadanie 2 - etap powiatowy}
\date{}
\author{Justyna Rojek-Nowosielska}

\begin{document}

\maketitle


\noindent \textbf{Wstęp} \\
Rozważmy następujący program, który wykonuje operacje na słowie podanym przez użytkownika. 
Zakładamy, że słowo składa się wyłącznie z wielkich liter alfabetu łacińskiego (A-Z).

\vspace{0.5cm}

% Kod programu
\begin{minipage}[t]{0.48\textwidth}
\textbf{C++}
\begin{lstlisting}[language=C++]
#include <iostream>
#include <string>
using namespace std;

int main() {
    string s;
    cin >> s;
    int n = s.length();
    int licznik = 0;

    for (int i = 0; i < n; i++) {
        // Porownanie znaku z jego
        // odbiciem lustrzanym
        if (s[i] == s[n - 1 - i]) {
            licznik++;
        }
    }

    cout << licznik << endl;
    
    if (licznik == n) {
        cout << "TAK";
    } else {
        cout << "NIE";
    }
    
    return 0;
}
\end{lstlisting}
\end{minipage}
\hfill

\vspace{0.5cm}
\hrule
\vspace{0.5cm}

\noindent Program przyjmuje na wejściu jeden ciąg znaków (napis).
\noindent Dla słowa KOT program wypisze NIE.

% Pytania
\pytanie{1.1. Co sprawdza warunek \texttt{licznik == n} na końcu programu?}
\begin{enumerate}[label=(\alph*), noitemsep]
    \item Czy słowo składa się z samych takich samych liter (np. "AAA"),
    \item Czy słowo jest palindromem (czyta się tak samo od przodu i od tyłu),
    \item Czy słowo ma parzystą długość,
    \item Czy słowo zawiera powtórzone znaki obok siebie (np. "ANNA").
\end{enumerate}

\pytanie{1.2. Co wypisze program dla wejścia: \texttt{KAJAK}?}

\pytanie{1.3. Co wypisze program dla wejścia: \texttt{MAMA}?}

\pytanie{1.4. Co wypisze program dla wejścia: \texttt{DOM}?}

\pytanie{1.5. Podaj przykład 4-literowego słowa, dla którego zmienna \texttt{licznik} przyjmie wartość 2.}

\pytanie{1.6. Ile razy wykona się pętla dla słowa o długości 10 znaków?}

\pytanie{1.7. Algorytm w obecnej formie jest mało wydajny. Wykonuje więcej porównań niż to konieczne. Ile razy minimalnie wystarczyłoby porównać znaki dla słowa o długości 10, aby program działał poprawnie?}

\pytanie{1.8. Jaka będzie wartość zmiennej \texttt{licznik} dla słowa \texttt{ABBA}?}

\pytanie{1.9. Użytkownik wpisał słowo \texttt{Informatyka}. Program wypisał \texttt{NIE}. Gdyby program miał ignorować wielkość liter (tzn. traktować 'a' i 'A' jako to samo), to jak należałoby zmodyfikować porównanie znaków?}

\pytanie{1.10. Analiza kodu.} \\
Zauważ, że pętla w programie przebiega od $0$ do $n-1$. Jeśli wprowadzimy słowo \texttt{AUTO}, porównania wyglądają tak:
\begin{itemize}[noitemsep]
    \item $i=0$: porównaj 'A' (s[0]) z 'O' (s[3]) $\rightarrow$ różne
    \item $i=1$: porównaj 'U' (s[1]) z 'T' (s[2]) $\rightarrow$ różne
    \item $i=2$: porównaj 'T' (s[2]) z 'U' (s[1]) $\rightarrow$ różne
    \item $i=3$: porównaj 'O' (s[3]) z 'A' (s[0]) $\rightarrow$ różne
\end{itemize}
Zmienna \texttt{licznik} wyniesie 0.
Podaj przykład słowa o długości 4, dla którego \texttt{licznik} wyniesie 0, ale słowo będzie zawierało dwie litery 'A'.

\end{document}
